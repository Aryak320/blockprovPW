\documentclass[10pt,a4paper]{scrartcl}
\usepackage[left=1cm, right=1cm]{geometry}
\usepackage{graphicx}
\usepackage{amssymb}
\usepackage{amsmath}
\usepackage{mathtools}
\usepackage{mathrsfs}
\usepackage{csquotes}
\usepackage{listings}
\usepackage{minted}
\usepackage[utf8]{inputenc}
\usepackage[bb=boondox]{mathalfa}
\usepackage{amsthm}
\newtheorem{theorem}{Theorem}[section]
\newtheorem{corollary}{Corollary}[theorem]
\newtheorem{lemma}[theorem]{Lemma}
\newtheorem*{remark}{Remark}
\theoremstyle{definition}
\newtheorem{definition}{Definition}[section]
\newtheorem{proposition}[theorem]{Proposition}
\theoremstyle{remark}
\usepackage[appendix=append]{apxproof}
\renewcommand{\appendixsectionformat}[2]{Semiring definitions for Section~#1 (#2)}
\DeclareMathAlphabet{\mathdutchcal}{U}{dutchcal}{m}{n}
\SetMathAlphabet{\mathdutchcal}{bold}{U}{dutchcal}{b}{n}
\DeclareMathAlphabet{\mathdutchbcal}{U}{dutchcal}{b}{n}
\newcommand{\dq}[1]{\text{\enquote{#1}}}
\newcommand{\op}{\hspace{0.1cm}\mathsf{op}\hspace{0.1cm}}
\makeatletter
\newcommand{\bigplus}{%
  \DOTSB\mathop{\mathpalette\mattos@bigplus\relax}\slimits@
}
\newcommand\mattos@bigplus[2]{%
  \vcenter{\hbox{%
    \sbox\z@{$#1\sum$}%
    \resizebox{!}{0.9\dimexpr\ht\z@+\dp\z@}{\raisebox{\depth}{$\m@th#1+$}}%
  }}%
  \vphantom{\sum}%
}
\newcommand{\mline}[1]{%
  \begin{multiline}
    #1
  \end{multiline}
}
\newcommand{\llbracket}{[\![}
\newcommand{\rrbracket}{]\!]}
\makeatother
\title{Block-level ( or Cluster-level) Possible-World Semantics in Relational Databases ( or Vector Databases)}
\author{Aryak 'Dawood' Sen \and Pratik 'Dawood' Karmakar}
\date{\today}



\begin{document}

\maketitle

\section{Introduction}

We present a possible-world semantics for provenance of aggregate comparison in relational databases and later try to extend it to vector databases. We leverage block-level abstractions to handle tuple-level uncertainty efficiently, using semantic clustering of variables.

\subsection{Semantic Partitioning of Variables}

Let $\mathbf{X}$ be the universe of variables (tuples). Fix a set of attribute(s) $A$ that define semantic similarity. Let
\[
\phi^A: \mathbf{X} \to \mathbb{R}^d
\]
be a deterministic feature/extraction map that transforms each tuple $\mathbf{x} \in \mathbf{X}$ into a $d$-dimensional representation suitable for vector search. Let
\[
g: \mathbb{R}^d \to \{1, \dots, m\}
\]
be a deterministic clustering method (or approximate nearest neighbor assignment). We define a map for assigning blocks from a finite collection $\mathcal{B}$ as
\[
\kappa: \mathbf{X} \to \mathcal{B} = \{B_1, \dots, B_m\}, \quad \kappa(\mathbf{x}) = B_{g(\phi^A(\mathbf{x}))}.
\]

The semantic partition induced by $\kappa$ is the family $\mathcal{B} = \{B_j\}_{j=1}^m$ with
\[
B_j = \{\mathbf{x} \in \mathbf{X} \mid \kappa(\mathbf{x}) = B_j \}.
\]

We require that $\{B_j\}$ forms a partition of $\mathbf{X}$, i.e.,
\[
\bigcup_{j=1}^{m} B_j = \mathbf{X}, \quad B_i \cap B_j = \emptyset \text{ for } i \neq j.
\]

\color{red}
\paragraph{Choices for $\phi^A$ and $g$ (Vector-database perspective)}

\begin{itemize}
    \item \textbf{Numeric attributes:} 
        \begin{itemize}
            \item $\phi(x)$: normalized numeric vector (z-score or min--max). 
            \item Similarity metric: Euclidean or Mahalanobis. 
            \item $g$: k-means, mini-batch k-means, or vector database indexing (e.g., Faiss, Annoy, HNSW).
        \end{itemize}
    \item \textbf{Categorical attributes:} 
        \begin{itemize}
            \item Encode with learned embeddings or frequency/one-hot $\to \phi(x)$.
            \item Similarity metric: cosine similarity.
            \item $g$: k-means on embedding space, hierarchical clustering, or ANN-based retrieval.
        \end{itemize}
    \item \textbf{Text attributes:} 
        \begin{itemize}
            \item Use sentence or attribute embeddings (e.g., Sentence-BERT, transformer CLS) $\to \phi(x)$.
            \item Similarity metric: cosine similarity.
            \item $g$: k-means, HDBSCAN, or ANN clustering in embedding space.
        \end{itemize}
    \item \textbf{Large-scale / approximate nearest neighbors:} 
        \begin{itemize}
            \item Use LSH, Faiss, or HNSW for embeddings; let $g$ assign LSH buckets or ANN clusters.
        \end{itemize}
    \item \textbf{Threshold-based semantics:} 
        \begin{itemize}
            \item Use DBSCAN with distance threshold $\varepsilon$ so that dense semantic neighborhoods become blocks; outliers form singletons.
        \end{itemize}
\end{itemize}
\color{black}

Ensure that the choice of metric and $g$ makes $\kappa$ deterministic (e.g., fixed initialization / random seed) to preserve reproducibility.

\subsection{Block-world mapping and lifted block-level possible-world semantics}

\begin{proposition}[Block-world mapping]
Let $\mathcal{B} = \{B_1,\dots,B_m\}$ be a collection of blocks disjointly partitioning $\mathbf{X}$ such that
\[
\bigcup_{k=1}^m B_k = \mathbf{X}, \quad B_i\cap B_j=\varnothing\ \text{for }i\neq j,
\]
and $\mathbf{X}_b:=\mathcal{B}$ is the analog of $\mathbf{X}$ but for blocks.

Define the mapping
\[
f:\mathcal{P}(\mathbf{X})\longrightarrow \mathcal{P}(\mathbf{X}_b),\qquad
f(W):=\{\,B\in\mathbf{X}_b \mid B\cap W\neq\varnothing\,\},
\]
which sends each tuple-level possible world $W\subseteq\mathbf{X}$ to the set $W_b\subseteq\mathbf{X}_b$ of blocks that contain at least one tuple of $W$.
\begin{enumerate}
  \item For every $W\subseteq\mathbf{X}$, the image $W_b=f(W)$ is unique.
  \item The mapping $f$ induces a partition of tuple-level worlds:
    for every $V\subseteq\mathbf{X}_b$,
    \[
      \mathcal{W}_V := \{\,W\subseteq\mathbf{X}\mid f(W)=V\,\},
    \]
    and the family $\{\mathcal{W}_V\}_{V\subseteq\mathbf{X}_b}$ partitions $\mathcal{P}(\mathbf{X})$.
  \item (Lifting of semantics) Let $(K,\oplus,\otimes,\mathbb{0}_K,\mathbb{1}_K,\delta,\ominus)$
    be the $m$-semiring of tuple-level semantics, $(M,+_M,\mathbb{0}_M)$ the additive monoid of aggregates, and $\mathdutchcal{u},\mathdutchcal{v}$ two aggregations. Define block-level annotations:
    \[
      \mathbf{x}_B^{(K)} = \bigoplus_{\mathbf{x}\in B} \mathbf{x}^{(K)}, \quad
      \mathbf{x}_B^{(M)} = \bigplus_{M, \mathbf{x}\in B} \mathbf{x}^{(M)},
    \]
    and induced aggregations $\widehat{\mathdutchcal{u}},\widehat{\mathdutchcal{v}}$ on $\mathbf{X}_b$. Then
    \[
      [\mathdutchcal{u}\oslash\mathdutchcal{v}]_{\mathrm{op}}
      =
      \bigoplus_{V\subseteq\mathbf{X}_b}
        \Bigl(
          \bigotimes_{B\in V\cap\mathrm{supp}(\widehat{\mathdutchcal{u}})} \mathbf{x}_B^{(K)}
          \otimes
          \bigotimes_{B\in V\cap\mathrm{supp}(\widehat{\mathdutchcal{v}})} \mathbf{x}_B^{(K)}
          \otimes
          \widehat{\chi}_{\mathrm{op}}\!\Bigl(\bigplus_{M,B\in V\cap\mathrm{supp}(\widehat{\mathdutchcal{u}})}\mathbf{x}_B^{(M)},
                                          \bigplus_{M,B\in V\cap\mathrm{supp}(\widehat{\mathdutchcal{v}})}\mathbf{x}_B^{(M)}\Bigr)
          \otimes
          \widehat{\Phi}_{V}
        \Bigr).
    \]
\end{enumerate}
\end{proposition}

 \begin{proof}
 \begin{enumerate}
 \item \textbf{Uniqueness of the image.}
 The mapping $f$ is defined deterministically: given a tuple-world $W\subseteq\mathbf{X}$,
 the set $f(W)=\{B\in\mathbf{X}_b\mid B\cap W\neq\varnothing\}$ is a well-defined subset of $\mathbf{X}_b$.
 Because $f$ is a function (each input $W$ is assigned exactly one output $f(W)$), the image $W_b$ is unique.

 \item \textbf{Fibres partition $\mathcal{P}(\mathbf{X})$.}
 For any $V\subseteq\mathbf{X}_b$ the fibre $\mathcal{W}_V=\{W\subseteq\mathbf{X}\mid f(W)=V\}$ collects exactly those tuple-worlds whose tuples lie precisely in the blocks indexed by $V$ (i.e. each block in $V$ contains at least one tuple of $W$ and no tuple of $W$ lies in a block outside $V$). Distinct $V$ produce disjoint fibres, and every tuple-world $W$ lies in the fibre of $f(W)$, hence the family $\{\mathcal{W}_V\}_{V\subseteq\mathbf{X}_b}$ is a partition of $\mathcal{P}(\mathbf{X})$.

 \item \textbf{Lifting of semantics.}
 Start from the tuple-level possible-world expansion from your semantics:
   \begin{align*}
   &[\mathdutchcal{u}\oslash\mathdutchcal{v}]_{\mathrm{op}}
 =
 \bigoplus_{W\subseteq\mathbf{X}}
   \Gamma(W),
     \\ &\text{where }\Gamma(W)
 \coloneqq
   \Bigl(\bigotimes_{W\cap\mathrm{supp}(\mathdutchcal{u})}\mathbf{x}^{(K)}\Bigr)
   \otimes
   \Bigl(\bigotimes_{W\cap\mathrm{supp}(\mathdutchcal{v})}\mathbf{x}^{(K)}\Bigr)
   \otimes
   \chi_{\mathrm{op}}\!\Bigl(\bigplus_{M,W\cap\mathrm{supp}(\mathdutchcal{u})}\mathbf{x}^{(M)},
                           \bigplus_{M,W\cap\mathrm{supp}(\mathdutchcal{v})}\mathbf{x}^{(M)}\Bigr)
   \otimes
   \Phi_{W}.
 \end{align*}
 Partition the outer sum according to the fibres of $f$:
 \[
 \bigoplus_{W\subseteq\mathbf{X}} \Gamma(W)
 =
 \bigoplus_{V\subseteq\mathbf{X}_b}\ \bigoplus_{W\in\mathcal{W}_V} \Gamma(W).
 \]
 This is merely a reorganization of the sum using the partition $\{\mathcal{W}_V\}$ established above.

 It remains to show that each inner sum $\bigoplus_{W\in\mathcal{W}_V}\Gamma(W)$ can be expressed in terms of block-level annotations
 $\mathbf{x}_B^{(K)}$ and $\mathbf{x}_B^{(M)}$. By the definitions
 \[
 \mathbf{x}_B^{(K)}=\bigoplus_{\mathbf{x}\in B}\mathbf{x}^{(K)}
 \qquad\text{and}\qquad
 \mathbf{x}_B^{(M)}=\bigplus_{M,\;\mathbf{x}\in B}\mathbf{x}^{(M)},
 \]
 the product over all tuple-level annotations in $W\cap\mathrm{supp}(\mathdutchcal{u})$
 can be regrouped block-wise. More precisely, because the blocks are disjoint,
 for $W\in\mathcal{W}_V$ the set $W\cap\mathrm{supp}(\mathdutchcal{u})$ is equal to
 \[
 \bigcup_{B\in V}\bigl(B\cap\mathrm{supp}(\mathdutchcal{u})\bigr),
 \]
 and therefore
 \[
 \bigotimes_{W\cap\mathrm{supp}(\mathdutchcal{u})}\mathbf{x}^{(K)}
 =
 \bigotimes_{B\in V}
   \Bigl(\bigotimes_{\mathbf{x}\in B\cap\mathrm{supp}(\mathdutchcal{u})}\mathbf{x}^{(K)}\Bigr).
 \]
 Applying the semiring addition $\oplus$ inside each block (which is how we define $\mathbf{x}_B^{(K)}$)
 and using distributivity in the $m$-semiring,
 the inner combination over all tuple-worlds $W$ mapping to the same block-world $V$ collapses to an expression
 involving only the block-level annotations and block-level aggregate values.
 Analogous regrouping applies to the $M$-sums used inside $\chi_{\mathrm{op}}$.

 Hence each inner sum $\bigoplus_{W\in\mathcal{W}_V}\Gamma(W)$ can be rewritten as a block-level term of the form
 \[
 \Bigl(\!\bigotimes_{B\in V\cap\mathrm{supp}(\widehat{\mathdutchcal{u}})} \mathbf{x}_B^{(K)}\Bigr)
 \otimes
 \Bigl(\!\bigotimes_{B\in V\cap\mathrm{supp}(\widehat{\mathdutchcal{v}})} \mathbf{x}_B^{(K)}\Bigr)
 \otimes
 \widehat{\chi}_{\mathrm{op}}\!\Bigl(\bigplus_{M,B\in V\cap\mathrm{supp}(\widehat{\mathdutchcal{u}})}\mathbf{x}_B^{(M)},
                                 \bigplus_{M,B\in V\cap\mathrm{supp}(\widehat{\mathdutchcal{v}})}\mathbf{x}_B^{(M)}\Bigr)
 \otimes
 \widehat{\Phi}_{V},
 \]
 where $\widehat{\chi}_{\mathrm{op}}$ tests the block-level aggregated quantities and $\widehat{\Phi}_{V}$
 is the block-level counterpart of the outside-world mask. Substituting these block-level terms back into the outer sum yields the claimed block-level expression.

 This completes the proof.

 \end{enumerate}
 \end{proof}

 \begin{remark}

 \begin{itemize}
   \item The partition (disjointness) hypothesis on the blocks is crucial for the simple form of the map $f$ and the algebraic regrouping. If blocks overlap (e.g. in certain storage/replication layouts) the mapping still exists but the algebraic lifting requires careful duplication-accounting.
   \item The equalities above are algebraic reorganisations inside the $m$-semiring and the monoid $(M,+_M)$. From an implementation viewpoint this is precisely why block-level possible-worlds can be used as a compact surrogate for tuple-level worlds, and why lazy expansion (reconstructing tuple-level annotations inside a block on demand) is sound: the block-level expression subsumes (by the fibres) the contributions of all tuple-level worlds lying under a given block-world.
 \end{itemize}
 \end{remark}


\end{document}
